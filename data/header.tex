\usepackage[a-1b]{pdfx}

\usepackage[T1]{fontenc}
\usepackage[utf8]{inputenc}
\usepackage[french]{babel}

\usepackage{libertine}
\usepackage{libertinust1math} % pour les maths
\usepackage{inconsolata} % pour le code

\usepackage{titlesec}
\titlelabel{\thetitle.\enspace}
\titleformat*{\section}{\scshape\fontseries{m}\selectfont\Large}
\titleformat*{\subsection}{\scshape\fontseries{m}\selectfont\large}
\titleformat*{\subsubsection}{\scshape\fontseries{m}\selectfont}

\usepackage{blindtext}
\usepackage{xcolor}
\usepackage{hyperref}
\usepackage{graphicx}
\usepackage[protrusion=true,expansion=true]{microtype}
\usepackage{amsmath, amssymb}
\usepackage{fancyhdr, fancyvrb}
\usepackage{geometry}
\usepackage{lastpage}
\usepackage{mathtools, amsthm}
\usepackage{listings}
\usepackage{eurosym}
\usepackage{multirow}
\usepackage{textcomp} % pour la commande textcomp \textquotesingle
\usepackage{array, colortbl}
\usepackage{longtable,booktabs}
\usepackage{tikz}
\usepackage{pgfplots}
\usetikzlibrary{positioning, backgrounds, fit}
\pgfplotsset{compat=newest}

\definecolor{myGrey}{RGB}{50,50,50}
\definecolor{myBlue}{RGB}{30,111,158}
\definecolor{myWaterGreen}{RGB}{30,158,111}
\definecolor{myBrown}{RGB}{158,111,30}
\definecolor{myPurple}{RGB}{158,30,111}
\definecolor{myViolet}{RGB}{111,30,158}
\definecolor{myRed}{RGB}{158,30,30}
\definecolor{myGreen}{RGB}{30,158,30}
\definecolor{myRose}{RGB}{240,30,128}
\definecolor{myGold}{RGB}{224,186,72}
\definecolor{myOrange}{RGB}{240,125,30}
\definecolor{myBackground}{RGB}{29,30,25}
\definecolor{myLightBackground}{RGB}{213,202,184}
\definecolor{myBeige}{RGB}{233,228,219}

\geometry{left=20mm, right=20mm, top=20mm, bottom=20mm}
\setlength{\parindent}{0pt}
\setlength{\parskip}{6pt plus 2pt minus 1pt}
\setlength{\headheight}{30.0pt}
\renewcommand{\headrulewidth}{0.5pt}
\renewcommand{\footrulewidth}{0.5pt}
\setlength{\columnseprule}{0.5pt}
\setlength{\emergencystretch}{3em}  % prevent overfull lines
\providecommand{\tightlist}{%
	\setlength{\itemsep}{0pt}
	\setlength{\parskip}{0pt}
}
\setcounter{tocdepth}{3}
\setcounter{secnumdepth}{5}

\newcommand{\passthrough}[1]{#1}

\arrayrulecolor{myLightBackground}

\pagestyle{fancy}
\fancyhf{}
\fancyhead[L]{\scriptsize\scshape\rightmark}
\fancyfoot[C]{page~\thepage~sur~\pageref{LastPage}}

\makeindex % On fait un index

\hypersetup{%
	pdfencoding=unicode,
	hypertexnames=false,
	pdfstartview=FitH,
	colorlinks=true,
	urlcolor=myBlue,
	linkcolor=myRed,
	citecolor=myGreen,
	bookmarksnumbered=true,
	bookmarksopen=true,
	bookmarksopenlevel=2,
	pdfborder={0 0 0}
}
