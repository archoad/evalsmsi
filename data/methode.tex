\section{Le SMSI}

\subsection{Exigences générales}

L'établissement doit établir, mettre en \oe{}uvre, exploiter, surveiller, réexaminer, tenir à jour et améliorer un SMSI documenté dans le contexte des activités d'ensemble de l'établissement et des risques auxquels elles sont confrontées. Le processus utilisé est basé sur le modèle PDCA.

\subsection{\'Etablissement et management du SMSI}

\subsubsection{\'Etablissement du SMSI}

L'organisme doit effectuer les tâches suivantes:

\begin{enumerate}
	\item définir le domaine d'application et les limites du SMSI en termes de caractéristiques de l'activité, de l'organisme, de son emplacement, de ses actifs, de sa technologie, ainsi que des détails et de la justification de toutes exclusions du domaine d'application;

	\item définir une politique pour le SMSI en termes de caractéristiques de l'activité, de l'organisme, de son emplacement, de ses actifs, et de sa technologie, qui:
	\begin{enumerate}
		\item inclut un cadre pour fixer les objectifs et indiquer une orientation générale et des principes d'action concernant la sécurité de l'information;
		\item tient compte des exigences liées à l'activité et des exigences légales ou réglementaires, ainsi que des obligations de sécurité contractuelles;
		\item s'aligne sur le contexte de management du risque stratégique auquel est exposé l'organisme, dans lequel se dérouleront l'établissement et la mise à jour du SMSI;
		\item établit les critères d'évaluation future du risque;
		\item a été approuvée par la direction.
	\end{enumerate}

	\item définir l'approche d'appréciation du risque de l'organisme:
	\begin{enumerate}
		\item identifier une méthodologie d'appréciation du risque adaptée au SMSI, ainsi qu'à la sécurité de l'information identifiée de l'organisme et aux exigences légales et réglementaires;
		\item développer des critères d'acceptation des risques et identifier les niveaux de risque acceptables. La méthodologie d'appréciation du risque choisie doit assurer que les appréciations du risque produisent des résultats comparables et reproductibles.
	\end{enumerate}

	\item identifier les risques:
	\begin{enumerate}
		\item identifier les actifs relevant du domaine d'application du SMSI, ainsi que leurs propriétaires\footnote{Le terme "propriétaire" identifie une personne ou une entité ayant accepté la responsabilité du contrôle de la production, de la mise au point, de la maintenance, de l’utilisation et de la protection des actifs. Ce terme ne signifie pas que la personne jouit à proprement parler de droits de propriété sur l'actif.};
		\item identifier les menaces auxquelles sont confrontés ces actifs;
		\item identifier les vulnérabilités qui pourraient être exploitées par les menaces;
		\item identifier les impacts que les pertes de confidentialité, d'intégrité et de disponibilité peuvent avoir sur les actifs;
	\end{enumerate}

	\item analyser et évaluer les risques c'est:
	\begin{enumerate}
		\item évaluer l'impact sur l'activité de l'organisme qui pourrait découler d'une défaillance de la sécurité, en tenant compte des conséquences d'une perte de confidentialité, intégrité ou disponibilité des actifs;
		\item évaluer la probabilité réaliste d'une défaillance de sécurité de cette nature au vu des menaces et des vulnérabilités prédominantes, des impacts associés à ces actifs et des mesures actuellement mises en \oe{}uvre;
		\item estimer les niveaux des risques;
		\item déterminer si les risques sont acceptables ou nécessitent un traitement, en utilisant les critères d'acceptation des risques;
	\end{enumerate}

	\item identifier et évaluer les choix de traitement des risques. Les actions possibles comprennent:
	\begin{enumerate}
		\item l'application de mesures appropriées;
		\item l'acceptation des risques en connaissance de cause et avec objectivité, dans la mesure où ils sont acceptables au regard des politiques de l'organisme et des critères d'acceptation des risques;
		\item l’évitement ou le refus des risques;
		\item le transfert des risques liés à l'activité associés, à des tiers, par exemple assureurs, fournisseurs;
	\end{enumerate}

	\item sélectionner les objectifs de sécurité et les mesures de sécurité proprement dites pour le traitement des risques.

	Les objectifs de sécurité et les mesures de sécurité proprement dites doivent être sélectionnés et mis en \oe{}uvre pour répondre aux exigences identifiées par le processus d'appréciation du risque et de traitement du risque. Cette sélection doit tenir compte des critères d'acceptation des risques ainsi que des exigences légales, réglementaires et contractuelles.

	Les objectifs de sécurité et les mesures de sécurité proprement dites doivent être sélectionnés comme partie intégrante de ce processus, dans la mesure où ils peuvent satisfaire à ces exigences.

	Les objectifs de sécurité et les mesures de sécurité proprement dites ne sont pas exhaustifs et des objectifs de sécurité et des mesures de sécurité proprement dites additionnels peuvent également être sélectionnés.

	Le questionnaire utilisé pour cette évaluation contient une liste complète d'objectifs de sécurité et des mesures de sécurité proprement dites qui se sont révélés communément appropriés aux organismes. Les utilisateurs peuvent se reporter à ce questionnaire comme point de départ de sélection des mesures de sécurité, afin de s'assurer qu'aucune option importante de sécurité n'est négligée.

	\item obtenir l'approbation par la direction des risques résiduels présentés;

	\item obtenir l'autorisation de la direction pour mettre en \oe{}uvre et exploiter le SMSI;

	\item préparer une déclaration d'acceptabilité (DdA). Une DdA\footnote{La DdA fournit un résumé des décisions concernant le traitement du risque. La justification des exclusions prévoit une contre-vérification qui permet d'assurer qu'aucune mesure n'a été omise par inadvertance} doit être élaborée et inclure les informations suivantes:
	\begin{enumerate}
		\item les objectifs de sécurité et les mesures de sécurité proprement dites et les raisons pour lesquelles ils ont été sélectionnés;
		\item les objectifs de sécurité et les mesures de sécurité proprement dites actuellement mis en \oe{}uvre;
		\item l'exclusion des objectifs de sécurité et des mesures de sécurité proprement dites spécifiés dans le questionnaire et la justification de leur exclusion.
	\end{enumerate}
\end{enumerate}

\subsubsection{Mise en \oe{}uvre et fonctionnement du SMSI}

L'organisme doit effectuer les tâches suivantes:

\begin{enumerate}
	\item élaborer un plan de traitement du risque qui identifie les actions à engager, les ressources, les responsabilités et les priorités appropriées pour le management des risques liés à la sécurité de l'information;

	\item mettre en œuvre le plan de traitement du risque pour atteindre les objectifs de sécurité identifiés, ce plan prévoyant le mode de financement et l'affectation de rôles et de responsabilités;

	\item mettre en œuvre les mesures de sécurité sélectionnées afin de répondre aux objectifs de sécurité;

	\item définir la méthode d'évaluation de l'efficacité des mesures ou groupes de mesures sélectionnés et spécifier comment ces évaluations doivent être utilisées pour évaluer l'efficacité des mesures, de manière à obtenir des résultats comparables et reproductibles.

	\item mettre en œuvre des programmes de formation et de sensibilisation;

	\item gérer les opérations du SMSI;

	\item gérer les ressources consacrées au SMSI;

	\item mettre en œuvre les procédures et les autres mesures permettant de détecter rapidement et de répondre tout aussi rapidement aux incidents de sécurité.
\end{enumerate}

\subsection{Surveillance et réexamen du SMSI}

L'organisme doit effectuer les tâches suivantes:

\begin{enumerate}
	\item exécuter les procédures de surveillance et de réexamen, ainsi que les autres mesures afin:
	\begin{enumerate}
		\item de détecter rapidement les erreurs dans les résultats des traitements;
		\item d'identifier rapidement les failles et les incidents de sécurité;
		\item de permettre à la direction de déterminer si les activités de sécurité confiées au personnel ou mises en œuvre par les technologies de l'information sont exécutées comme prévu;
		\item de faciliter la détection des événements de sécurité, et par conséquent, de prévenir les incidents de sécurité par l'utilisation d'indicateurs;
		\item de déterminer si les actions entreprises pour résoudre une faille de sécurité se sont révélées efficaces.
	\end{enumerate}

	\item réaliser des réexamens réguliers de l'efficacité du SMSI (y compris le respect de la politique et des objectifs du SMSI, et le réexamen des mesures de sécurité) en tenant compte des résultats des audits de sécurité, des incidents, des mesures de l'efficacité, des propositions et du retour d'information de toutes les parties intéressées;

	\item d’évaluer l'efficacité des mesures afin de vérifier que les exigences de sécurité ont été satisfaites;

	\item réexaminer les appréciations du risque à intervalles planifiés et réexaminer le niveau de risque résiduel et le niveau de risque acceptable identifié, compte tenu des changements apportés à l'organisme, à la technologie, aux objectifs métiers et aux processus de l'organisme, aux menaces identifiées, à l'efficacité des mesures \oe{}uvre et aux évènements extérieurs (modifications apportées à la législation ou à la réglementation, aux obligations sontractuelles et au climat social);

	\item mener des audits internes du SMSI à intervalles fixés;

	\item effectuer une revue de direction du SMSI de manière régulière afin de s'assurer du caractère toujours adéquat du domaine d'application du système et de l'identification des améliorations apportées au processus d'application du SMSI;

	\item mettre à jour les plans de sécurité afin de tenir compte des résultats des activités de surveillance et de réexamen;

	\item consigner les actions et les événements qui pourraient avoir un impact sur l'efficacité ou les performances du SMSI.

\end{enumerate}


\subsection{Mise à jour et amélioration du SMSI}

L'organisme doit effectuer les tâches suivantes de manière régulière:

\begin{enumerate}
	\item mettre en œuvre les améliorations identifiées du SMSI;

	\item entreprendre les actions correctives et préventives appropriées. Appliquer les leçons tirées des expériences de sécurité des autres organismes, ainsi que de celles de l'organisme concerné;

	\item informer toutes les parties prenantes des actions et améliorations, avec un niveau de détail approprié aux circonstances et, le cas échéant, convenir de la méthode à adopter;

	\item s'assurer que les améliorations permettent d'atteindre leurs objectifs prévus.
\end{enumerate}



\clearpage
